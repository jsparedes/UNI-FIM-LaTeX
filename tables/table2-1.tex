% -*- coding: utf-8; -*-
\begin{table}[H]
\centering
\begin{tabular}{l>{\raggedright}p{4in}} \toprule
Elemento & \centering{}Descripción\tabularnewline\midrule
$C$ & es el conjunto de todas las configuraciones posibles del entorno,
es decir, los obstáculos y las configuraciones libres.\tabularnewline
$C_{free}$ & Subconjunto de C, configuraciones libres de obstáculos existentes.\tabularnewline
$R$ & Indicador de ponderación de proximidad al punto deseado. La distancia
euclidiana es la más utilizada.\tabularnewline
$q_{init}$ & Configuración inicial\tabularnewline
$q_{fin}$ & Configuración que se desea alcanzar\tabularnewline
$q_{rand}$ & Configuración aleatoria dentro del espacio $C_{free}$ \tabularnewline
$q_{near}$ & Es la conguración mas proxima a $q_{rand}$ , en el sentido denido
por R, de entre las existentes en un árbol. Se evalúa con el indicador
de ponderación.\tabularnewline
$q_{new}$ & Configuración a añadir al árbol.\tabularnewline
$e$ & Longitud de segmento de crecimiento. Geométricamente, es la distancia
entre un punto del árbol y el siguiente con el que esta conectado.\tabularnewline
$Arbol$ & Estructura de datos.\tabularnewline
\bottomrule
\end{tabular}

\caption{Elementos de una RRT}


\label{Flo:cnomen}
\end{table}
