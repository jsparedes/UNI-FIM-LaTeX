% -*- coding: utf-8; -*-
\chapter{ALGORITMO RRT}


\section{Elementos}



El algoritmo RRT original se basa en la construcción de un árbol de
configuraciones que crece buscando a partir de un punto origen. Para
entender el algoritmo se usará la nomenclatura expresada en el cuadro


% -*- coding: utf-8; -*-
\begin{table}[H]
\centering
\begin{tabular}{l>{\raggedright}p{4in}} \toprule
Elemento & \centering{}Descripción\tabularnewline\midrule
$C$ & es el conjunto de todas las configuraciones posibles del entorno,
es decir, los obstáculos y las configuraciones libres.\tabularnewline
$C_{free}$ & Subconjunto de C, configuraciones libres de obstáculos existentes.\tabularnewline
$R$ & Indicador de ponderación de proximidad al punto deseado. La distancia
euclidiana es la más utilizada.\tabularnewline
$q_{init}$ & Configuración inicial\tabularnewline
$q_{fin}$ & Configuración que se desea alcanzar\tabularnewline
$q_{rand}$ & Configuración aleatoria dentro del espacio $C_{free}$ \tabularnewline
$q_{near}$ & Es la conguración mas proxima a $q_{rand}$ , en el sentido denido
por R, de entre las existentes en un árbol. Se evalúa con el indicador
de ponderación.\tabularnewline
$q_{new}$ & Configuración a añadir al árbol.\tabularnewline
$e$ & Longitud de segmento de crecimiento. Geométricamente, es la distancia
entre un punto del árbol y el siguiente con el que esta conectado.\tabularnewline
$Arbol$ & Estructura de datos.\tabularnewline
\bottomrule
\end{tabular}

\caption{Elementos de una RRT}


\label{Flo:cnomen}
\end{table}



\newpage{}


\section{Pseudocódigo}

El algoritmo viene expresado por:

\medskip{}


%
\begin{algorithm}[H]
\begin{lyxcode}
\textbf{Función:~RRT(}$\mathbf{q}_{\mathbf{ini}}$\textbf{,}~$\mathbf{K{}_{max}}$\textbf{,~$e$)}

1.~$Arbol[0]$~=~$q_{init}$;

2.\textbf{~for}~$K=1$~\textbf{to}~$K{}_{max}$

3.~$q_{rand}$~=~Configuración\_Aleatoria();

4.~$q_{near}$=~Elemento\_más\_cercano~($Arbol$,~$q_{rand}$)

5.~$q_{new}$=~Nuevo\_elemento~($q_{rand}$,$q_{near}$,~$e$)

6.\textbf{~end~for}

7.~Devuelve~$Arbol$;~\medskip{}


\textbf{Función:~Nuevo\_elemento}($q_{rand}$,$q_{near}$,~$e$)

1.~$u_{qnear-qrand}=\left(q_{rand}-q_{near}\right)/\left\Vert q_{rand}-q_{near}\right\Vert $;

2.\textbf{~}$q_{new}=q_{near}+e.u_{qnear-qrand}$;

3.~Devuelve~$q_{new}$;
\end{lyxcode}
\caption{Algoritmo RRT}

\end{algorithm}


\bigskip{}


El pseudocódigo de RRT es explicado como sigue:


\section{Algoritmos evolucionados de RRT}


\subsection{RRT Bidireccional Básica}


\subsection{RRT Ext-Ext}


\subsection{RRT Ext-Con}


\section{Sistemas no holónomos}

\selectlanguage{english}%
